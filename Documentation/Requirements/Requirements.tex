\documentclass{article}

\usepackage[a4paper,left=18mm,right=18mm,top=20mm,bottom=18mm]{geometry}
\usepackage[italian]{babel}

\usepackage{titling}
\usepackage{graphicx}
\usepackage{subcaption}
\usepackage{float}

\title{Requirements for Software engineering project}
\author{Nicolas Anselmi, David Guzman Piedrahita and Marco Vinciguerra}
\date{\today}


\begin{document}
\maketitle

\section{Introduction}

\subsection{Purpose} 
Questo documento segue la struttura proposta dallo standard IEEE 830 per la definizione dei requirements. Questi rappresenta uno dei criteri fondamentali per valutare l’adeguatezza dell’implementazione ottenuta nei diversi sprint e un punto di partenza per le seguenti fasi dello sviluppo dell’applicazione.
\subsection{Scope}
L’obiettivo centrale è costruire un’applicazione per la gestione di prenotazioni di tagli in saloni di bellezza, eventualmente estendibile ad altri servizi. 
\\In grandi linee, l’applicazione dovrebbe consentire di facilitare il processo della creazione di appuntamenti, evitando l’uso di chiamate, sostituendole con procedure online molto più veloci e friction-less. 
\\L’interfacciamento dell’utente con l’applicazione deve essere tale che, nonostante la più alta complessità inerente a una soluzione model-view-controller rispetto all’uso di semplici chiamate, l’utilizzatore percepisca un netto miglioramento rispetto al solito modo di fare e a prescindere delle loro conoscenze informatiche. 
\\Di conseguenza, l’applicazione non solo deve offrire la funzionalità centrale delle prenotazioni, ma in più deve farlo in un modo intuitivo. 

\subsection {Definitions, acronyms and abbreviations }
\subsection {References} 
\subsection {Overview} 
Dopo questa introduzione, il documento è composto da due ulteriori fasi: la parte 2, che da una descrizione globale dei requirements, e la parte 3, che elenca e classifica tutti i diversi requirements individuali, usando il modello Kano e Moscow.
\section {Overall description} 
\subsection {Product perspective} 
Non ci sono database o altre strutture informatiche preesistenti, in quanto il pubblico di destinazione è composto proprio da saloni di bellezza che gestiscono le loro prenotazioni in modo informale. La costruzione del software deve dunque partire da zero.
\\ Per la fase di elicitation sono state usate le strategie di open-ended interview, task analysis derivante anche da esperienze personali e natural language descriptions. 
\\ Le successive fasi di V and V e di negotiation saranno svolte dopo la generazione di un primo prototipo.
\subsection {Product functions} 
A continuazione sono elencati le funzionalità centrali, l’elenco dei requirements a una granularità più precisa sono disponibili nella parte 3.
\begin{itemize}
\item Possibilità di prenotare tagli e, eventualmente, altri servizi offerti dai saloni.
\item Possibilità di visualizzare le prenotazioni future e passate associate al salone in questione.
\item Possibilità di tener traccia dei clienti con degli appositi profili utente/cliente associati a informazioni rilevanti.
\item Disponibilità di visualizzare l’andamento dei ricavi risultanti dalle prenotazioni. 
\end{itemize}
\subsection {User characteristics} 
Come detto sopra, il prodotto deve essere costruito per soddisfare le esigenze di un pubblico di destinazione con conoscenze tecniche basilari (uso di smartphone e applicazioni user-friendly). 
\\L’UI dell’applicazione deve essere facile da usare da utenti ormai familiarizzatisi con altre applicazioni molto popolari (YouTube, Instagram…), di conseguenza il design-language deve essere coerente con quello al quale i potenziali utenti si sono ormai abituati, riducendo, di conseguenza, la learning-curve per usare il software.

\subsection {Constraints} 
Nella sua versione finale, l’applicazione dovrebbe essere utilizzabile sia da cliente che prenotano che da gestori di saloni di bellezza. 
\\I clienti dei saloni non devono avere accesso a informazioni del salone, come le prenotazioni di altri utenti, o i ricavi del salone. Dall’altro canto, i gestori non possono modificare certe informazioni dei profili degli utenti, ma possono modificare dati associati alle prenotazioni. 
\\Per il prototyping iniziale, si deve dare priorità alle funzionalità offerte ai gestori. L’implementazione delle prenotazioni remote fatte direttamente dagli utenti ha una priorità secondaria nelle prime fasi.
\subsection {Assumptions and dependencies}
\subsection {Requirements subsets} 
\section {Specific requirements}

\subsection{Kano model}
\subsubsection{Attractive:}
\begin{itemize}
    \item Possibilità di scegliere i diversi tipi di acconciature e in 
    base al tipo di acconciatura scegliere la durata della prenotazione 
\end{itemize}
\subsubsection{Must-be:}
\begin{itemize}
    \item Possibilità di registrarsi per la prima volta al sito da parte di un cliente.
    \item Utilizzo di un database per la gestione dei clienti. In alternativa si potrebbero utilizzare 
    variabili per gestire le prenotazioni.
    \item Creazione di un profilo base utente per le prenotazioni.
\end{itemize}

\subsubsection{One-Dimensional}
\begin{itemize}
    \item Che il sistema dellle prenotazioni non funzioni correttamente 
    e che si accavallino sulla stessa fascia oraria.
\end{itemize}

\subsubsection{indifferent:}
\begin{itemize}
    \item Troppa personalizzazione dell'utente (immagini + acconciature consigliate).
\end{itemize}
\subsubsection{Reverse:}

\subsubsection{Questionable:}
\begin{itemize}
    \item Possibilità di mettere recensioni da parte dell'utente.
\end{itemize}

\subsection{MoSCoW}

\subsection{Must haves}
\begin{itemize}
    \item Possibilità di scegliere i diversi tipi di acconciature e in 
    base al tipo di acconciatura scegliere la durata della prenotazione 
\end{itemize}

\subsection{Should haves}
\begin{itemize}
    \item Possibilità di registrarsi per la prima volta al sito da parte di un cliente.
    \item Utilizzo di un database per la gestione dei clienti. In alternativa si potrebbero utilizzare 
    variabili per gestire le prenotazioni.
    \item Creazione di un profilo base utente per le prenotazioni.
    \item Che il sistema dellle prenotazioni non funzioni correttamente 
    e che si accavallino sulla stessa fascia oraria.
\end{itemize}

\subsection{Could haves}
\begin{itemize}
    \item Possibilità di mettere recensioni da parte dell'utente.
\end{itemize}

\subsection{Won't haves}


\end{document}
