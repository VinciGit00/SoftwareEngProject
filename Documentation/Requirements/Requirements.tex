\documentclass{article}

\usepackage[a4paper,left=18mm,right=18mm,top=20mm,bottom=18mm]{geometry}
\usepackage[italian]{babel}

\usepackage{titling}
\usepackage{graphicx}
\usepackage{subcaption}
\usepackage{float}

\title{Requirements for Software engineering project}
\author{Nicolas Anselmi, David Guzman Piedrahita and Marco Vinciguerra}
\date{\today}


\begin{document}
\maketitle
\section{Kano model}
\subsection{Attractive:}
\begin{itemize}
    \item Possibilità di scegliere i diversi tipi di acconciature e in 
    base al tipo di acconciatura scegliere la durata della prenotazione 
\end{itemize}
\subsection{Must-be:}
\begin{itemize}
    \item Possibilità di registrarsi per la prima volta al sito da parte di un cliente.
    \item Utilizzo di un database per la gestione dei clienti. In alternativa si potrebbero utilizzare 
    variabili per gestire le prenotazioni.
    \item Creazione di un profilo base utente per le prenotazioni.
\end{itemize}

\subsection{One-Dimensional}
\begin{itemize}
    \item Che il sistema dellle prenotazioni non funzioni correttamente 
    e che si accavallino sulla stessa fascia oraria.
\end{itemize}

\subsection{indifferent:}
\begin{itemize}
    \item Troppa personalizzazione dell'utente (immagini + acconciature consigliate).
\end{itemize}
\subsection{Reverse:}

\subsection{Questionable:}
\begin{itemize}
    \item Possibilità di mettere recensioni da parte dell'utente.
\end{itemize}

\section{MoSCoW}

\subsection{Must haves}
\subsection{Should haves}
\subsection{Could haves}
\subsection{Won't haves}


\end{document}
