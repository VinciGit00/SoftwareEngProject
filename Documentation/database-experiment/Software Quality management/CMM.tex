\documentclass{article}

\usepackage[a4paper,left=18mm,right=18mm,top=20mm,bottom=18mm]{geometry}
\usepackage[italian]{babel}

\usepackage{titling}
\usepackage{graphicx}
\usepackage{subcaption}
\usepackage{float}
 
\title{CMM - Capability Maturity Model}
\author{Nicolas Anselmi, David Guzman Piedrahita and Marco Vinciguerra}

\begin{document}
\maketitle
Per quanto riguarda il CMM (Capaibility Maturity Model) esistono 5 livelli di maturità del software, essi sono:
\begin{itemize}
    \item Initial
    \item Managed
    \item Defined
    \item Measurable
    \item Optimization
\end{itemize}
In questo progetto si punta ad utilizzare il terzo livello 3 (Defined) in cui ogni attività viene documentata
e standardadizzata per l'intera organizzazione del processo per il design, development e testing.


\end{document}
