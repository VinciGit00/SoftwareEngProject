\documentclass{article}

\usepackage[a4paper,left=18mm,right=18mm,top=20mm,bottom=18mm]{geometry}
\usepackage[italian]{babel}

\usepackage{titling}
\usepackage{graphicx}
\usepackage{subcaption}
\usepackage{float}

\title{Life-Cycle: SCRUM  (sprint 2)}
\author{Nicolas Anselmi, David Guzman Piedrahita and Marco Vinciguerra}


\begin{document}
\maketitle

Il life cyle utilizzato è Scrum. Di conseguenza, saranno usate iterazioni con limiti di tempo predefinito, i cosiddetti sprint. La loro composizione e quantità è soggetta a modifiche.
La durata tradizionale degli sprint e di 2 a 4 settimane, ma in questo caso si utilizzano sprint di durata da 5 a 10 giorni.


\subsection{Scrum Roles}
\subsubsection{Product owner}
\subsubsection{Development team}
\subsubsection{ScrumMaster}

\subsection{}{Sprints} 
\subsubsection{Sprint 1}
Il primo sprint comprende la costruzione dei primi requirements e la prima versione dell’architettura e design, insieme a un’implementazione basilare del codice vero e proprio, di conseguenza è prevista una quantità di giorni superiore a quella che si prevede per succesivi sprint. Inoltre, la fase di maintenance ha un ruolo secondario nel primo sprint, e la fase di testing è più interessata a definire come e quando fare testing, e meno alla sua completa messa in pratica: l’approccio di queste ultime due fasi cambierà nei prossimi sprint, i quali partiranno da basi più solide. 
\subsubsection{State of the backlog}
il backlog iniziale si basa completamente sui requisiti che sono disponibile nel corrispettivo documento, nella fase di architettura il backlog tiene considerazione dei suddetti requisiti ma anche del contesto architettonico (vedi documento architettura) . 
\subsubsection{Meetings}
\begin{itemize}
\item Sprint planning meeting 
\\26/12/21 
\\Sono stati definiti gli obiettivi base del progetto e dello sprint. Dal backlog (e, a sua volta, dai requirements) sono stati scelti i seguenti elementi per lo sprint backlog:
\\-	Creazione della documentazione dei requirements.
\\-	Creazione della documentazione di architettura.
\\-	Proposte iniziali di schemi UML (soggetti a molte modifiche dovuto al seguente punto).
\\-	Precisazione delle librerie e framework elements da usare per l’implementazione software.
\\-	Processo cots per la scelta delle suddette librerie e altri elementi di framework.

\item Scrum meetings
\\28/12/21 
\\Discussione sui primi schemi UML. Proposte per migliorarli e modificarli.
\\30/12/21
\\Discussione su come gestire nel modo più efficiente la costruzione della UI.
\\2/12/22
\\Discussione di diversi elementi modulari UI trovati nei giorni precedenti e che erano candidati per uso come UI ufficiale. 
\\5/12/22
\\Discussione sul come costruire la home page dell’applicazione.
\\6/12/22
Cross-review dei diversi file di documentazione creati (requirements, architecture, testing, ecc) 
\item Sprint review meeting 
\item Retrospective meeting notes
\item Backlog refinement
\end{itemize}

\end{document}
