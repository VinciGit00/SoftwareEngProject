\documentclass{article}

\usepackage[a4paper,left=18mm,right=18mm,top=20mm,bottom=18mm]{geometry}
\usepackage[italian]{babel}

\usepackage{titling}
\usepackage{graphicx}
\usepackage{subcaption}
\usepackage{float}

\title{Software architecture specification (sprint 1)}
\author{Nicolas Anselmi, David Guzman Piedrahita and Marco Vinciguerra}


\begin{document}
\maketitle

\section{Rappresentazione architettura con IEEE 1471} 
Le rappresentazioni più concrete dell'architettura dell'applicativo vengono descritte attraverso degli schemi UML. Questi possono essere categorizzati come delle View indirizzate a diversi Stakeholder e definite tramite diversi Viewpoint, come formalizzato dallo standard IEEE 1471.
\\In particolare, i Viewpoint utilizzati per definire le View sono stati scelti tra quelli proposti da Bass et al. (2003).
\subsection{Module View}
I Viewpoint nella categoria module costituiscono una rappresentazione statica del sistema. Per questo progetto la module view più importante è Class. 
\\Le view costruite seguendo le specifiche del viewpoint Class descrivono il sistema in termini delle relazioni di eredità degli elementi. Il class diagram UML del progetto mette a disposizione questa informazione, assieme ad altre precisazioni dovute alla natura object-oriented del linguaggio di programmazione. 
\\\textbf{Add image}
\subsection{Component and connector View}
Il tipo di viewpoint scelto in questa categoria è il process viewpoint. Questo definisce il sistema in termini di comunicazione e sincronizzazione di processi. 
\\Come tutti gli altri viewpoint in questa categoria, offre una descrizione dinamica del software e, nel caso particolare del process viewpoint, è particolarmente utile per valutare performance e availability del sistema. 
\\In termini di UML, questa informazione è disponibile principalmente tramite il Sequence diagram. Questo schema spiega come i diversi elementi (classi) del sistema comunicano tra di loro attraverso procedure calls \textbf{(di tipo sincrono e asincrono)}.
\subsection{Allocation View}
Questa categoria di viewpoint descrive la relazione tra il sistema e il mondo che lo circonda. Di conseguenza, può essere utilizzato per definire come viene assegnato hardware al software o come quest’ultimo viene mappato al file system. 
\\Per il progetto in questione, questo tipo di problematiche vengono gestite automaticamente dai framework utilizzati e quindi non sono competenza degli stakeholder che altrimenti ne farebbe uso, come i programmatori e i maintaner.
\\I viewpoint di tipo work assignment potrebbero essere utile per rappresentare graficamente la distribuzione del lavoro, ma in questo sprint vengono saltati per brevità. 


\end{document}
