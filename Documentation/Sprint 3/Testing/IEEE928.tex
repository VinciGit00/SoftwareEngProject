\documentclass{article}

\usepackage[a4paper,left=18mm,right=18mm,top=20mm,bottom=18mm]{geometry}
\usepackage[italian]{babel}

\usepackage{titling}
\usepackage{graphicx}
\usepackage{subcaption}
\usepackage{float}
\usepackage{hyperref}

\title{IEEE 829}
\author{Nicolas Anselmi, David Guzman Piedrahita and Marco Vinciguerra}

\begin{document}
\maketitle
\section{Introduzione}
Questo tipo di documento si occupa di specificare e documentare come avviene la fase di testing
del progetto. Questo documento rappresenta anche il test plan. Siccome stiamo facendo un tipo di
sviluppo di software AGILE i test non vengono fatti alla fine ma vengono svolti in contemporanea allo
sviluppo delle classi.
\\Viene svolto un tipo di sviluppo di tipo TDD in cui si scrivono le classi dei test in 
contemporanea (se non prima) delle classi che servono per il sito. Le fasi del testing sono le 
seguenti:
\begin{itemize}
    \item Preparation of tests 
    \item Running the  tests
    \item Completion of testing
\end{itemize}

\section{Preparation of tests}
I test vengono fatti in contemporanea allo sviluppo delle classi, talvolta per alcune classi
sono stati fatti prima i test pensando poi alle classi che sarebbero state implementate successivamente.
I test per le classi in dart vengono scritti anche loro in linguaggio dart ed eseguiti tramite Junit.
\\Per quanto riguarda il testing di flutter si utilizza la funzione testWidgets che si importa 
dal pacchetto $flutter_test$ e permette di fare i test sulla web app.
\\I test tenderanno ad essere il più possibile di tipo coverage in quanto si cercerà di coprire 
il più possibile i casi di test.
\\All'interno del progetto flutter vengono utilizzate 3 classi differenti che si occupano dello 
sviluppo del testing, essi sono:
\begin{itemize}
    \item $UnitTest.dart$: per fare unit testing
    \item $WidgetTest.dart$: per fare i test delle widget
\end{itemize}

\section{Running the tests}
Una volta fatte le classi ed eseguiti i test si procede in modo iterativo a modificare il codice
finchè i test non danno tutti risultati positivi

\section{Completion of testing}
I dati expected sono inventati e ad ogni iterazione si commentano i risultati per capire 
se va tutto bene e se c'è qualcosa che non quadra.

\end{document}
