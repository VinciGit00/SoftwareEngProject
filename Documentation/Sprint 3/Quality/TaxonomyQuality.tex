\documentclass{article}

\usepackage[a4paper,left=18mm,right=18mm,top=20mm,bottom=18mm]{geometry}
\usepackage[italian]{babel}

\usepackage{titling}
\usepackage{graphicx}
\usepackage{subcaption}
\usepackage{float}

\title{Taxonomy of qualities attributes}
\author{Nicolas Anselmi, David Guzman Piedrahita and Marco Vinciguerra}

\begin{document}
\maketitle
\subsection{Introduzione alla Taxonomy quality}
Questo file si occupa del descrivere i requisiti tassonomici di qualità del progetto.
\\Per quanto riguarda la definizione di McCall.
\\In particolare si vuole utiizare i seguenti driver/linee guifa per la produzione, revisione e
transizione  del codice.
\section{Product operation}
\begin{itemize}
    \item Correctness: se il sistema fa quello richiesto
    \item Reliability: se il sistema è abbastanza accurato
    \item Efficiency: se il sistema utilizza l'hardware efficientemente
    \item Integrity: se il sistema è sicuro
    \item Usability: se il sistema è utilizzabile
\end{itemize}
In particolare ci si sofferma sull'usability, correctness e reliability in quanto il 
prodotto deve funzionare il meglio possibile.

\section{Product revision}
\begin{itemize}
    \item Maintainability: se il sistema in caso di guasto è riparabile
    \item Testability: se il sistema è testabile 
    \item Felxibility: se il sistema è facilmente cambiabile
\end{itemize}
Il prodotto che si sta costruendo in questo caso tende a essere molto testabile in 
quanto i test per verificare la correttezza vengono creati ed eseguiti poco dopo la  creazione
delle classi.

\section{Product transition}
\begin{itemize}
    \item Portability: se il software è utilizzabile su altre piattaforme
    \item Reusability: se il software è riutilizzabile
    \item Interoperability: se il sistema è interfacciabile con altri sistemi
\end{itemize}
Il sistema tenderà a essere molto portabile in quanto è stato fatto co Flutter, il quale
tende ad essere molto scalabile.

\end{document}

