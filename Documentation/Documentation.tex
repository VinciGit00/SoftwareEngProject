\documentclass{article}

\usepackage[a4paper,left=18mm,right=18mm,top=20mm,bottom=18mm]{geometry}
\usepackage[italian]{babel}

\usepackage{titling}
\usepackage{graphicx}
\usepackage{subcaption}
\usepackage{float}

\title{Documentation for Software engineering project}
\author{Nicolas Anselmi, David Guzman Piedrahita and Marco Vinciguerra}
\date{\today}


\begin{document}
\maketitle

\section{Introduction}
Il progetto prevede lo sviluppo di una mobile app per gestire la prenotazione di 
un negozio di parruccheria. C'è la possibilità di avere due tipi di utente:
un cliente e un padrone di negozio. Il cliente ha la possibilità di prenotare diversi 
tipi di acconciatura direttamente senza interfacciarsi /chiamare direttamente il proprietario
del negozio. Ogni tipologia di taglio selezionabile ha una durata e può consentire un orario customizzabile 
da parte del cliente.
I membri del team sono: Nicolas Anselmi, David Guzman Piedrahita e Marco Vinciguerra.

\section{Process model}
Il life cycle del progetto è agile, in particolare la tecnica utilizzata è SCRUM con
sprint di 5 giorni in quanto il tempo per la consegna è imminente. Il periodo di sviluppo parte
poco prima di Natale. Ogni giorno verso le 9 30 
c'è un daily scrum tenuto dallo scrum master in cui si discutono le problematiche riscontrate durante 
il giorno precedente e le possibili soluzioni a queste.

\section{Organization of the projcet}

\section{Standards, guidelines, procedures}
I principali linguaggi di programmazione del progetto saranno: python e dart, quest'ultimo
viene esteso tramite flutter.
Si usa il vocabolario standard di flutter.

\section{Management activities}

\section{Risks}
Il rischio principale è di non consegnare in tempo il progetto.
\section{Staffing}
I membri del team sono: Nicolas Anselmi, David Guzman Piedrahita e Marco Vinciguerra.
\\Sono 3 studenti di ingegneria del terzo anno.

\section{Methods and techiniques}

\section{Quality assurance}

\section{Work packages}


\section{Resources}

\section{Budget and schedule}

\section{Changes}

\section{Delivery}
La consegna verrà fatta 5 giorni prima dell'esame orale.
\end{document}
